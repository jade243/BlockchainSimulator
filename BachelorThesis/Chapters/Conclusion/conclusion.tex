In this work, we've studied blockchains operation, then we focused our analysis on the mining process and we implemented a blockchain simulator to analyze and understand this process better. Our analysis was divided is three main chapters:\newline

First, as mining is based on SHA256, we studied the threats involved in case SHA256 is compromised. However, the probabilities of such a situation are low and there are contingency plans to counter the problem. \newline

Then, we've studied the robustness of the system by exploring some techniques like selfish mining or how to double-spend coins. Like the previous argument, the probabilities of success are low and would involve a lot of resources like computers, electrical power, … \newline

Finally, we tried to have a step ahead by looking at quantum technology. We discovered that it could represent an important threat to mining and blockchains' security. The challenge, to take advantage of this technology, is now to build a stable and powerful enough quantum computer and it may require a decade to reach this point. \newline

To conclude on the security of the mining process, the different threats that we've studied are quite unlikely or not ready yet to be set up. Moreover, the security measures like the contingency plans allow us to trust this technology and explain its growing popularity.
