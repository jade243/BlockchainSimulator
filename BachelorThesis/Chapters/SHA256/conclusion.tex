\section{Conclusion}

  \subsection{Consequences summary}

We can sum up the different consequences of breakages on SHA256. \newline

\begin{tabular}{ll}

  \underline{\textbf{Breakage}} & \underline{\textbf{Consequences}} \\
  Pre-image & Complete breakage of the blockchain \\
  Bounded pre-image & Complete breakage of the blockchain \\
  Second pre-image & Double spending, steal coins \\
  Collision & Double spending, steal coins \\

\end{tabular}

  \subsection{Contingency plans}

All the attacks presented above are based on potential SHA256 vulnerabilities. We cannot guarantee that SHA256 will stay safe forever. However, we can notice it was created in 2001 and no significant weaknesses have been discovered yet so we can conclude SHA256 is quite robust (see a StackExchange conversation about SHA256 security, \cite{SHA256_security}). \newline

In case SHA256 is broken, Bitcoin has a contingency plan (see \cite{contingency}).

As we've seen this situation would be dramatic, an attacker could compromise the whole system, this includes the alert system.

The plan in this situation is to ask the users to shut down their clients and to hardcode the public keys of all addresses that have unspent outputs. Then, these keys will be used when a new version of the blockchain is released. \newline

The code for all of this should be prepared but, in reality, this is not the case because, even if the consequences would be severe, the risk is very low.
